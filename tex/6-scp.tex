\section{SCP:一种联邦拜占庭协商协议}\label{sec:scp}

本节给出了恒星共识协议(Stellar Consensus Protocol,SCP)。高层次上来讲,SCP包含两个子协议:一个提名协议和一个表决协议。表决协议产生为槽点的候选值。如果运行时间足够长,它将在每个完好节点处产生相同的候选值集合,这意味着节点可以用一种确定性的方式把候选值组合起来。然而这有两大注意事项。首先,节点无法知道提名协议何时达到收敛点。其次,即使在收敛之后,恶性行为节点可能有能力重置提名过程多次。

当节点猜测到提名协议已经收敛后,它们执行表决协议,这采用了联邦选举来提交或终止与合成值相关联的表决。当节点同意提交一个表决时,和表决相关联的值将为讨论中的槽点具体化。当它们同意终止一个表决时,表决的值变得不再相关。如果一个表决在某个状态下被卡住了——在这种情形下一个或多个完好节点不能够提交或终止表决,那么节点重新使用更高的表决来尝试;他们将新的表决和被卡住的表决一样的值相关联,以防止任何节点错以为被卡住的节点被提交了。直观上来说,安全性来源于所有被卡住的或已提交的表决和相同的值相关联。存活性这一事实保证:一个被卡住的表决可以通过转向更高的表决来中立化。

本节的剩余部分给出了提名协议和表决协议。它们各自先以概念性语句描述,然后用带有表示概念性语句集合的消息的具体协议描述。最终,第\ref{sec:scp_correct}说明了协议的正确性。SCP认为每个槽点是完全独立的且可被看成是一个单一槽点共识协议中的许多分开的实例(和Paxos算法~\cite{Lamport:1998:PP:279227.279229}中的``单一法令议会''\footnote{single-decree synod。}类似)。我们必须总是在一个特定的槽的环境下解释例如候选值和表决等概念,而不像替他的讨论中隐式表述这样一个槽。

\subsection{提名协议}

\subsubsection{具体的提名协议}
\subsection{表决协议}
一旦节点有了合成值它们将参与表决协议,尽管提名会继续更新合成值。一个表决$b$一个形如$b=\langle n,x\rangle$的二元组,这里$x\neq \perp$是一个值,而$b$是对讨论中的槽具体化的请示书(referendum)。$n\geq 1$是一个确保大些的表决数总是可访问的计数器。我们使用类C语言的标记$b.n$和$b.x$来表示表决$b$的计数和值的域,因而有$b=\langle b.n, b.x\rangle$。表决是全序的,而$b.n$比$b.x$更为重要{\footnote{译注:指二元组$b_1\prec b_2$当且仅当$b_1.n < b_2.n$或$b_1.n=b_2.n $且$b_1.x < b_2.x$;$b_1\equiv b_2$当且仅当$b_1.n=b_2.n\cap b_1.x=b_2.x$。}}为了方便起见,一个特殊无效的空表决$\mybm{0}=\langle 0,\perp\rangle$小于其他任何表决,而一个特别的计数器$\infty$大于其他所有的计数器。

我们分别用提交或终止一个表决$b$作为使用联邦投票来对语句$commib\;b$和$abort\;b$进行认可。对于给定的表决,$commit$和$abort$是相互冲突的,因此一个良性行为的节点最多为它们中的一个投赞成票。在第\ref{sec:voting}节的标注系统下,$commit\;b$的反是$\overline{commit\;b}$,但这里使用$abort\;b$更为直观。

由于对某个槽至多只有一个值被选用,所有提交的和被卡住的表决必须包含相同的值。粗略地说,这意味着如果陈述$commit$和更小的非终止表决相冲突的话那么它是无效的。

\begin{definition}[相容的]
        两个表决是\textbf{相容的}(记作$b_1 \sim b_2$)当且仅当$b_1.x=b_2.x$;它们是\textbf{不相容的}(记作$b_1\not\sim b_2$)当且仅当$b_1.x\neq b_2.x$。我们还将$b_1\leq b_2$(或等价地,$b_2\geq b_1$)且$b_1\sim b_2$记作$b_1\lesssim b_2$($b_2\gtrsim b_1$)。类似地,$b_1\lnsim b_2$或$b_2\gnsim b_1$意味着$b_1\leq b_2$(或等价地$b_2\geq \b_1$)且$b_1\not\sim b_2$。
\end{definition}

\begin{definition}[就绪的]
        一个表决$b$是\textbf{就绪的}当且仅当下面集合中的每个陈述都是正确的:$\left\{abort\;b_{old}|b_{old}\lnsim b\right\}$。
\end{definition}

更准确地说,如果$b$被确认是就绪的话那$commit\;b$对投赞成票来说是有效的,节点通过在对应的终止陈述的联邦投票来保障它。全体一致地对这些陈述进行投票是方便的,因此不论我们在哪里写了``$b$就绪''周围的环境将应用于$abort$陈述的整个集合中。特别地,一个节点投票赞成、接受或确认$b$就绪当且仅当它分别投票赞成、接受或确认它们全部\textit{终止}了。

为了提交一个表决并向外界展示它的值$b.x$,SCP节点首先接受并确认$b$已经就绪,然后接受并确认$commit\;b$。在第一个完好节点投票赞成$commit\;b$之前,经由联邦投票的准备步骤确保所有完好节点最终可以确认$b$是就绪的。当一个完好节点$v$接受$commit\;b$时,意味着$b.x$最终将会被选中。然而,正如第\ref{sec:voting_safety}中所讨论的那样,为了防止$v$被污染$v$必须在作用于它之前确认$commit$。

\subsubsection{具体的表决协议}\label{sec:scp_ballot_concrete}

\todo{图16}强调了由每个节点维护的每一{\slot}的状态。一个节点$v$存储了:它当前的表决$b$;两个最近的已经认定就绪的且不相容的表决对$(p,p^{\prime})$;它必须投票\textit{提交}的(或在后续阶段需要确认\textit{提交}的)最小表决$c$(如果存在的话),对此它还没有接着接受到\textit{终止类}陈述;已确认就绪的最高表决$P$;从每个节点($M$)处接受到的最新消息;以及状态$\varphi$。表决$b$,$p$,$p^{\prime}$和$P$在同一个阶段里是不减的。另外,如果$c\neq\mybm{0}$——意味着$v$可能参与了批准$commit\;c$——代码必须确保$c\lesssim P\lesssim b$。这一不变量保证了节点总是可以投票为当前的表决$b$做好准备。

\todo{图17}展示了协议消息。注意$a\;\vee accept(a)$是每个节点需要为一个{\quorum}所断言的,使得它们按照\textit{接受}定义中的第\ref{itm:cond_normal}种方式接受$a$。每个节点通过设置$b\leftarrow \langle 1,combine(Z)\rangle$,$p\leftarrow \mybm{0}$,$p^{\prime}\leftarrow \mybm{0}$,$P\leftarrow \mybm{0}$,$c\leftarrow \mybm{0}$,$M\leftarrow\emptyset$及$\varphi \leftarrow \textsl{PREPARE}$的方式初始化{\slot}的状态。之后节点在同类间重复地交换消息,发送由 $\varphi$表明的任何消息。一旦给$M$添加了一个新近接受的消息,一个节点$v$按照下面的方式添加它的状态:

\begin{enumerate}\label{protocal_case}
	\item 如果$\varphi = \textsl{PREPARE}$且接受的信息让$v$接受新表决是就绪的,更新$p$和$p^{\prime}$。之后,如果$c\neq \mybm{0}$且$p\gnsim P$或$p^{\prime}\gnsim P$,设置$c\leftarrow \mybm{0}$。
	\item 如果$\varphi = \textsl{PREPARE}$且$v$确认新表决是就绪的,增加$P$。之后,如果$c=\mybm{0}$,$P\geq b$,且$p\gnsim P$或$p^{\prime}\gnsim P$都不成立,则设$c\leftarrow P$且$b\leftarrow P$(尽管通常$b=P$已经成立)。
	\item 如果$\varphi = \textsl{PREPARE}$,且$v$接受一个或多个相容表决的\textit{提交类}消息。设$c$为最小的这类表决,$P$设为最大的使得``$v$能够接受所有的$\left\{commit\;b^{\prime}|c\lesssim b^{\prime}\lesssim P\right\}$,$b\leftarrow \langle \infty, c.x\rangle$和$\varphi\leftarrow \textsl{CONFIRM}$''的表决。
	\item 如果$\varphi = \textsl{CONFIRM}$,且接受的消息让$v$接受新表决为就绪的,则提升$p$至最高已被接受为就绪态的、且满足$p\sim c$的表决。
	\item 如果$\varphi = \textsl{CONFIRM}$且$v$接受更多的相容的\textit{提交类}消息,提升$p$至最高的``使得$v$接受所有的$\left\{commit\;b^{\prime}|c\lesssim b^{\prime} \lesssim P\right\}$''的表决。
	\item 如果$\varphi = \textsl{CONFIRM}$且$v$对任意$c^{\prime}$确认$commit\;c^{\prime}$,设置$c$和$P$为最低和最高的这类表决,设$\varphi\leftarrow\textsl{EXTERNALIZE}$,具体化$c.x$并结束。
\end{enumerate}

当$c=\mybm{0}$时,上述协议实施联邦选举来确认$b$已经就绪。一旦$c\neq \mybm{0}$,该协议对$commit\;c$ (实际上是介于$c$和$P$之间的相容的表决)实施联邦选举。对确认阶段来说,一旦一个良性行为的节点$v$接受了$commit\;c$,该节点就不会接受或尝试确认任何满足$c^{\prime}\not\sim c$的$commit\;c^{\prime}$。因此直观上说,一旦一个\textit{提交}被确认了,只要节点具有{\quorum}交属性那么具体化它的值就是安全的。

所有来自一个节点的消息在元组$\langle \varphi,b,p,p^{\prime},P\rangle$的定义之下是全序的,这里$\varphi$是最重要的域而$P$最不重要。所有的\textsl{PREPARE}消息都在\textsl{CONFIRM}消息之前,转而对于给定的{\slot}来说在单独的\textsl{EXTERNALIZE}消息之前。\textsl{PREPARE}信息显式地包含这四个域,而\textsl{CONFIRM}和\textsl{EXTERNALIZE}包含\todo{图17}中所描述的值。这一序关系使得$M$值包含来自每个节点的最新表决而不依赖于时间来排序消息成为可能,这是因为网络环境可能会对消息重新排序。

一些协议的细节需要解释。形如``$abort\;b^{\prime}\vee accpet(abort\;b^{\prime})$''的由\textsl{PREPARE}所蕴含的陈述并没有指明$v$是否赞成或确认$abort\;b^{\prime}$。对于\textit{接受}的定义来说这种区分并不重要。掩盖这种区分使得$v$忘记了旧的它投票提交(因此不能够投票终止)的表决--只要它为这些表决接受一个\textit{终止类}消息的话。

为了确保节点收敛于$P$,$p$和$p^{\prime}$都是必需的,这是因为定理\ref{thm:confirmed_stats_keep_liveness}要求节点重新广播它们已经接受的消息。从\textit{就绪}的定义可知,``一个节点接受为就绪态的、两个不相容的最高表决''蕴含了``所有该节点接受为就绪态的表决''。

在$v$发出\textsl{ENTERNALIZE}消息的时候它实际上已经接受了一个区间内的\textit{提交类}消息$\left\{commit\;b^{\prime}|b^{\prime}\gtrsim c\right\}$。然而,$v$设置$P$来断言``只有它确认提交的表决是可接受的'',而不是在隐式的\textsl{CONFIRM}消息中设置$P.n=\infty$从而对每个$b^{\prime}\gtrsim c$断言$accept(commit\;b^{\prime})$。这样做是足够的,因为一旦一个单独的完好节点确认了$commit\;c$,定理\ref{thm:confirmed_stats_keep_liveness}告诉我们所有的完好节点也将确认它。把关注点集中在已被确认的表决上有额外的好处:\textsl{EXTERNALIZE}消息仅断言$v$已经批准的信息,从而使得$\mybm{Q}(v)$不再相关。这意味着一个独立静态的\textsl{EXTERNALIZE}消息对未来任意远处的想赶上进度的节点来说都是有用的,即使{\quorum}切片与此同时已经改变了很多。

交换表决消息只需要一个RPC。参数是发送者最新的消息而返回值是接受者最新的消息。对于\textsl{NOMINATE},如果$D$或在表决中的值$x$是加密哈希,那么为了取回没有被缓存的哈希原像需要一个单独的RPC。
\subsubsection{表决选择}
\subsection{正确性}\label{sec:scp_correct}

一个节点只有在已经许诺确认所有小编号的表决的\textit{终止}陈述之后才能担保确认$commit\;b$陈述。因为一个良性行为的节点不能够接受(因此也不担保确认)相冲突的陈述,这意味着对于给定的$\mybm{V},\mybm{Q}$,定理\ref{th5}确保一个良性行为节点集合$S$只要享有除$\mybm{V}\backslash S${\quorum}可交性则不会产生相互冲突的值。如果$\mybm{V}$和$\mybm{Q}$只在槽间改变的话那么安全性仍然成立,但如果它们在槽中(mid-slot)改变呢(例如用于应对节点崩溃)?为了分析在重新配置的情形下的安全性,我们保守地对旧的和新的{\quorum}切片集合进行交操作;这反映了这样一个事实:节点可能依据来自不同时期的消息的组合来作出决定。因此很保守地讲,一个节点只有在当前槽用到的每个配置下都是完好的我们才说它是完好节点。但是我们可以放松要求而说:如果一个节点在最近的配置中都是完好的并且在以往的配置中从未接受过来自全部由恶性行为节点的{\vblock}集合发来的消息,则我们称该节点是完好的。

\begin{theorem}\label{th12}
	令$\langle \mybm{V_1},\mybm{Q_1}\rangle,\ldots,\langle \mybm{V_k},\mybm{Q_k}\rangle$是一个FBAS在协商一个单独槽点的时候经历过的配置集合。令$\mybm{V}=\mybm{V_1}\cup \cdots\cup \mybm{V_k}$且$\mybm{Q}(v)=\left\{q|\exists j, v\in\mybm{V_j}\cap q\in\mybm{Q}_j(v)\right\}$。令$B\subseteq\mybm{V}$是一个集合,满足$B$包含所有已经发送了非法消息的恶性行为节点---尽管$\mybm{Q}\backslash B$可能仍然包含崩溃(不响应)的节点。假设$v_1\not\in B$具体化了$x_1$,而$v_2\not\in B$具体化了$x_2$。则如果$\langle\mybm{V},\mybm{Q}\rangle^{B}$有{\quorum}交,那么$x_1=x_2$。
\end{theorem}

\begin{proof}
	为了让$v_1$产生具体的$x_1$,它必然已经和一个伪{\quorum}$U_1\subseteq\mybm{V}$合作批准了$accept(commit(\langle n_1,x_1\rangle))$。我们称伪{\quorum}是因为$U_1$对任意特殊的$j$来讲可能都不是$\langle\mybm{V}_j,\mybm{Q}_j\rangle$的{\quorum},这是由于批准可能已经涉及包含多个配置的消息。然而,为了使得批准成功,$\forall v\in U_1,\exists j, \exists q\in\mybm{Q}_j(v)$使得$q\subseteq U_1$。从$\mybm{Q}$的构造方式可知$q\in\mybm{Q}$。因此$U_1$是$\langle\mybm{V},\mybm{Q}\rangle$。类似地,一个伪{\quorum}必然已经批准了$accept(commit\langle n_2,x_2\rangle)$,且$U_2$一定是$\langle\mybm{V},\mybm{Q}\rangle$的一个{\quorum}。根据$\langle\mybm{V},\mybm{Q}\rangle^{B}$的{\quorum}交,必然存在某个$v\in \mybm{V}\backslash B$使得$v\in U_1\cap U_2$。根据假设,这样的$v\not\in B$不能声明接受不相容的表决。由于$v$确认接受值为$x_1$和$x_2$的表决提交,那么必然有$x_1=x_2$。
\end{proof}

对于节点$v$的存活性,当一个FBAS对一个单独节点经历了一系列的重新配置$\langle \mybm{V_1},\mybm{Q_1}\rangle,\ldots,\langle \mybm{V_k},\mybm{Q_k}\rangle$时我们需要考虑一些东西。首先,定理\ref{th12}的安全性前提条件必须对$v$以及$v$关心的节点成立,因为违反安全性会破坏定理\ref{th10}中所要求的{\quorum}交属性。其次,最新状态下的恶性行为节点集合$\langle \mybm{V_k}, \mybm{Q_k}$必须不能是{\vblock}的,这是因为这会否定一个{\quorum}而阻止它批准陈述。最后,$v$的状态不能够被一个{\vblock}集合错误地声称接受$\langle \mybm{V_1},\mybm{Q_1}\rangle\cdots \langle \mybm{V_{k-1},\mybm{Q_{k-1}}\rangle}$的中一个陈述所污染。总之,我们认为节点$v$在下面的条件成立的情况下是完好的。首先,在恶性行为节点被删除时对当前槽的过去的配置的并必须有{\quorum}交。其次,$v$必须在最新的视图$\langle \mybm{V_k}, \mybm{Q_k}$依据以往的静态配置标准还是完好的。最后,$v$必须从未接受过以往的配置$\langle \mybm{V_1},\mybm{Q_1}\rangle\cdots \langle \mybm{V_{k-1},\mybm{Q_{k-1}}\rangle}$中的全有恶性节点组成的{\vblock}集合中接受过消息。