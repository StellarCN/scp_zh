\subsection{例子和讨论}
\todo{图3}展示了一个层状系统,系统中的不同节点有着截然不同的切片集合,这可能只有在FBA才能实现。顶层由$v_1,\ldots,v_4$组成,其结构类似于$PBFT$中$f=1$的情形,这意味着只要其他三个节点可达并正常工作,它可以容忍一个拜占庭故障。节点$v_5,\ldots,v_8$组成中间层并且不相互依赖,而是依赖于顶层。中间层的节点只要求两个顶层节点就可以形成切片。(假定顶层只有最多一个拜占庭故障,那么除非整个系统出错否则两个顶层节点不会同时出现故障。)节点$v_9$和$v_10$在叶子层,其切片由任意两个中间层节点组成。注意这里$v_9$和$v_10$可能选择不相交的切片集合,例如$\left\{v_5,v_6\right\}$和$\left\{v_7,v_8\right\}$;然而,两者都会间接依赖于顶层节点。

实际情形中,顶层可能包含来自各个地方的从四到几十个广为人知并可信的金融机构。当顶层的大小增长时,可能不再有关于它的成员关系的准确认可,但是将会有顶层成员间意识上很大程度的重叠。另外,我们可以想象很多的中间层,例如每个代表一个国家或者地理区域。

这种分层系统和域名间的网络路由系统十分类似。当今的网络是由独立的对等直连和网络对间的传输关系共同组成的。没有中央权威机构来指派或仲裁这些安排。然而这些成对的关系也已足够创建出实际意义上的第一层结构——网络服务提供商(ISP)~\cite{peer_isp2010}。	尽管英特网的可达性受防火墙影响,但传递性的可达性几乎是完好的---例如,某个防火墙可能会阻塞纽约时报,但如果它允许Google访问,而Google能够访问纽约时报,那么纽约时报也间传递性地可达。传递性可达对网站来说或许是受限的设施,但是这对共识至关重要;等价的例子是Google仅当在纽约时报接受某个陈述的时候才接受该陈述。

如果我们把{\quorum}切片看成类似网络可达性,而把{\quorum}看成是传输性可达,那么网络的几乎完全的传输性可达暗示我们同样也可以利用FBA达成世界范围内的共识。在很多方面,共识比网际间的路由转换要容易得多。传输消耗资源并且花费资金,但包含切片的过程仅仅要求检查数字签名。因此,FBA节点可以在包含切片的那端报告错误,相比较常见的在对等直连和传输安排见到的那样,这可以用更为相互依赖且冗余的方式来建立保守的切片。

另一个对中心化网络来说不可能实现完成的是有环依赖结构,例如\todo{图4}中展示的那样。这样一个环状结构不会被有意生成出来,然而当独立的嗯节点选择他们自己的切片时,这可能导致整个网络最终被植入了环状依赖。更大的问题是,相比较传统的拜占庭一致性来说,一个FBA的协议必须解决远为多样的{\quorum}结构。