\documentclass[a4paper,onecolumn,12pt]{article}
%\documentclass[prodmode,acmtocs]{acmsmall}

\usepackage{graphicx}
\usepackage{amsmath, amsfonts, amssymb}
\usepackage{bm}
\usepackage[xetex]{hyperref}
%\usepackage[all]{hypcap}

%\usepackage{glossaries}
%\makeglossaries

% Not ACM compliant, but much better looking
% \usepackage[notext]{stix} 
% \def\UrlFont{\ttfamily}
% \renewcommand{\ttdefault}{txtt}

\usepackage{booktabs,colortbl,tabu}
\usepackage{xspace}

%%%%%%%%%%%%%%%%%%%%
\usepackage{xcolor}
\definecolor{few-gray-bright}{HTML}{010202}
\definecolor{few-red-bright}{HTML}{EE2E2F}
\definecolor{few-green-bright}{HTML}{008C48}
\definecolor{few-blue-bright}{HTML}{185AA9}
\definecolor{few-orange-bright}{HTML}{F47D23}
\definecolor{few-purple-bright}{HTML}{662C91}
\definecolor{few-brown-bright}{HTML}{A21D21}
\definecolor{few-pink-bright}{HTML}{B43894}

\definecolor{few-gray}{HTML}{737373}
\definecolor{few-red}{HTML}{F15A60}
\definecolor{few-green}{HTML}{7AC36A}
\definecolor{few-blue}{HTML}{5A9BD4}
\definecolor{few-orange}{HTML}{FAA75B}
\definecolor{few-purple}{HTML}{9E67AB}
\definecolor{few-brown}{HTML}{CE7058}
\definecolor{few-pink}{HTML}{D77FB4}

\definecolor{few-gray-light}{HTML}{CCCCCC}
\definecolor{few-red-light}{HTML}{F2AFAD}
\definecolor{few-green-light}{HTML}{D9E4AA}
\definecolor{few-blue-light}{HTML}{B8D2EC}
\definecolor{few-orange-light}{HTML}{F3D1B0}
\definecolor{few-purple-light}{HTML}{D5B2D4}
\definecolor{few-brown-light}{HTML}{DDB9A9}
\definecolor{few-pink-light}{HTML}{EBC0DA}

\newcommand{\maincolor}{orange}
\taburulecolor{few-\maincolor-bright}
%%%%%%%%%%%%%%%%%%%%

\usepackage{fontspec}
\usepackage[slantfont, boldfont]{xeCJK} % 允许斜体和粗体
% CJK字体
\setCJKmonofont[Path=fonts/]{SimHei.ttf} % 等宽
\setCJKsansfont[Path=fonts/,BoldFont={SimHei.ttf},ItalicFont={SimKai.ttf}]{SimSun.ttf} % 无衬线
\setCJKmainfont[Path=fonts/,BoldFont={SimHei.ttf},ItalicFont={SimKai.ttf}]{SimSun.ttf} % 衬线
% 英文字体
\setmainfont[Mapping=tex-text,Path=fonts/]{LiberationSerif-Regular.ttf} % 衬线
\setsansfont[Mapping=tex-text,Path=fonts/]{LiberationSans-Regular.ttf} % 无衬线
\setmonofont[Mapping=tex-text,Path=fonts/]{LiberationMono-Regular.ttf} % 等宽
\punctstyle{kaiming} % 开明式标点格式
\usepackage{indentfirst} % 首段缩进
\setlength{\parindent}{2em}

\hypersetup{colorlinks, linkcolor=black,
  filecolor=black, urlcolor=black, citecolor=blue, pdftitle={恒星共识协议:互联网级共识的联邦模型}, pdfauthor={David Mazi{\`e}res}, pdfsubject={联邦拜占庭协议}, pdfkeywords={FBA, SCP}, pdfproducer={xelatex} }

\newtheorem{theorem}{定理}
%\newtheorem*{theorem*}{定理}
\newtheorem{lemma}{引理}[section]
%\newtheorem*{lemma*}{引理}[section]
\newtheorem{definition}{定义}[section]
%\newtheorem*{definition*}{定义}[section]
\newtheorem{proof}{证明}

\newcommand{\todo}[1]{{\footnotesize \textcolor{red}{$\ll$\textsf{TODO #1}$\gg$}}}%
\newcommand{\mybm}[1]{{\bm #1}}%
\newcommand{\quorum}{群体}
\newcommand{\slot}{存储单元}
\newcommand{\vblock}{$v-\!\!$阻塞}

\renewcommand{\figurename}{图}
\renewcommand\refname{参考文献}
\renewcommand{\abstractname}{摘\hspace{2em}要}

\newcommand{\FBA}{SCP\@\xspace}

\title{恒星共识协议:互联网级共识的联邦模型}
\author{David Mazi{\`e}res\\恒星发展基金会}
\date{}

\begin{document}
\maketitle

\abstract{}
本文介绍了一种新型的共识方法——联邦拜占庭一致协议(Federated Byzantine Agreement, FBA)。FBA的健壮性来源于{\quorum}切片,即由单个节点的个体信任决策共同决定系统级别的仲裁。这些切片联结系统的做法和当今独立网络对等直连、转发决定统一化互联网的方式非常相似。

我们还提出恒星共识协议(Stellar Consensus Protocol, SCP),一种FBA的构造方法。和所有拜占庭一致性协议一样,SCP不做任何关于攻击者理性行为的假设。此前的拜占庭一致性模型预设一个全体一致接受的成员名单;和它们不同,SCP允许开放的成员关系,这促进了有机网络的成长。相比于去中心化的工作量证明(PoW)及权益证明(PoS)机制,SCP对计算能力以及经济成本消耗要求适度,降低了进入门槛并潜在地把金融系统开放给新的参与者。
\section{介绍}

目前的金融基础设施由一堆混乱的封闭系统组成。这些系统之间的差异不仅造成了交易费用偏高,同时也导致了跨政治和地理边界的资产转移缓慢。这种摩擦制约了金融服务的发展,使得数十亿人所受金融服务不足。

为了解决这些问题,我们需要建立一种金融技术设施,让它支持我们从互联网那里看到的有机发展和创新的特色,同时仍然保持金融交易的完整性(integrity)。历史上,我们依赖于高进入门槛以保证完整性。我们相信已经建立起来的金融机构并尽力去规范化他们的行为。然而这种排他性和有机增长的目标矛盾。增长需要新的创新型参与者,而他们可能只拥有不多的财力和计算资源。

我们需要一个向任何人开放的世界性金融网络,这样新的参与者可以加入并扩展未享有服务社群的金融渠道。构建这种网络的挑战在于确保参与者准确记录交易。由于进入门槛较低,用户毋需相信服务提供商来监管他们自己。由于范围遍及全世界,提供商也毋需相信某个单一实体来运营网络。一种令人瞩目的方案是建立去中心化的系统:参与者通过对另一参与者的交易有效性达成共识来确保完整性。这种协定取决于世界范围的共识机制。

本文提出了联邦拜占庭协议FBA——一个适用于世界范围的共识的模型。在FBA中,每个参与者知道它认为重要的其它成员。任何交易在参与者认为已结算之前将等待其它参与者中的绝大多数认可该交易。反之,那些重要的参与者将不认可这笔交易,直到有他们认为重要的其它参与者也认可,以此类推。最终,网络中有足够的节点接受某笔交易,这使得它不能被撤销。直到那时任何参与者认为本次交易结算完成。FBA的共识可以确保金融网络的完整性。它的去中心化控制可以激励有机增长。

本文进一步提出了恒星共识协议(SCP), 一种FBA的构造方法。我们证明SCP的安全性对于一个异步的协议是最优的,这是因为它确保了在承认这样该保障的任何节点错误情景下的一致性。我们还说明了,除非参与节点出现错误使得信任依赖无法满足, 否则SCP不会引发阻塞状态——在这些状态中不可能存在共识。SCP是第一个同时满足下述四个关键属性且被证明安全的共识机制:

\begin{itemize}
	\item \textbf{去中心化控制。} 任何人都可以参与,并且不需要中心化权威机构认定``对共识来说谁的认可是必须的''。
	\item \textbf{低延迟。} 实际情形中,节点可以在人们对互联网或者支付交易所期待的时间范围内(即最多几秒)达成共识。
	\item \textbf{灵活的信任。} 用户有信任他们认为合适的任意团体组合的自由。例如,一个小的非营利组织可能在维持大型机构诚实性方面起关键作用。
	\item \textbf{渐进安全性。} 安全性取决于数字签名和哈希函数族,其参数被切实调节到对抗具有难以想像超强计算能力的对手。
\end{itemize}

在金融市场之外,SCP也有确保组织诚实履行关键职能的应用。一个好例子是数字证书认证机构(CA),可认为他们持有互联网网络的钥匙。经验表明CA在少数场合签署了不正确的认证~\cite{ac_ms2013,dc_google2015} 。一些研究~\cite{Kim:2013up,ct_google2013,Basin:2014bn,cryptoeprint:2014:1004} 建议通过认证透明性(Certificate Transparency)解决这个问题。SCP可以增强认证透明性:不仅任何个体只要愿意就可以简单地审计CA行为,同时审计者还能够认可所有已颁发的认证全体,这使得退回或者重写之前颁发的认证而不被发现变得困难。

下一节讨论以往的共识方法。第\ref{sec:fba}节定义了联邦拜占庭一致性(FBA),同时阐述了这一系统希望达到的安全性和存活性的概念。第\ref{sec:resilience}节讨论FBA系统中的最优化错误恢复,由此为SCP建立了安全目标。第\ref{sec:voting}节构建联邦选举机制,这是SCP协议中关键的一个构建环节。第\ref{sec:scp}节阐述了SCP本身,并证明在满足信任依赖的情况下它的安全性以及不会导致阻塞状态。第\ref{sec:limit}节讨论SCP的局限性。最后,第\ref{sec:summary}节概述结论。对于不熟悉数学符号的读者,附录\ref{sec:glossary}定义了文中使用的一些符号。
\section{相关工作}

\todo{图1}总结了SCP和以往的共识机制的区别。最有名的去中心化共识机制是比特币提出的工作量证明(Proof-of-Work,PoW)模式。比特币采用两方面措施来达成共识。首先,它为理性参与者提供了激励机制希望他们表现良好。其次,它通过工作量证明的算法来结算交易,这一设计保护系统免受没有绝大多数计算能力恶意行为者的影响。比特币强烈地表明了人们去中心化共识的迫切希求。

然而工作量证明方式是存在局限的。首先,它浪费资源:通过2014年一次估计,比特币消耗的电力或许和爱尔兰整个国家的消耗相当。第二,安全交易结算需要忍受几分钟甚至几十分钟的预期延时。最后,和传统的加密协议不同,工作量证明算法不提供渐进安全。考虑非理性攻击者或者破坏共识的外在激励,少量的计算优势可以使得安全假设失效,导致交易历史被所谓的``51\%''攻击重写。更糟的是,最初拥有少于50\%算力的攻击者可以利用系统向加入他们的成员提供不成比例的奖励。作为拥有最大算力支持的标杆数字货币,比特币拥有一定程度面对51\%攻击的保护能力。稍小一点的系统则已经成了牺牲品~\cite{attack_bbt2013,attack_cb2013},暴露了任何不基于比特币区块链的工作量证明系统的一个问题。

工作量证明的一个替代方案是权益证明(Proof-of-State,PoS),这种系统中的共识依赖提供抵押的各方。类似于工作量证明,奖励鼓励理性参与者遵守协议;有些系统还惩罚作恶~\cite{slasher2014, neucoin2015} 。权益证明使得所谓的``零成本''攻击成为可能,其中之前提供抵押但后来兑换并花掉的各方可以回来重写他们仍然拥有权益时的历史。为了减轻这种攻击的危害,系统把工作量证明结合到权益证明——正比于权益降低所需工作量——或者延缓抵押归还足够长时间直到其它(有时是非正式的)共识机制建立不可逆转的检查点。

还有另一个共识方式是拜占庭协商~\cite{Pease:1980:RAP:322186.322188,Lamport:1982:BGP:357172.357176},其最知名的变形是PBFT~\cite{Castro:1999:PBFT}。拜占庭协商在存在一部分参与者的随意(包括非理性)行为的情况下可以确保共识。这方式有两个令人瞩目的属性。首先,共识能够快速而且高效。第二,信任和资源所有权解耦,这使得小的非营利机构帮助更强力的组织比如银行或者认证中心保持诚实成为可能。然而麻烦的是各方必须在精确的参与者列表上达成一致。还有必须防止攻击者多次加入系统或超出系统容错范围,即所谓的``女巫攻击''~\cite{Douceur:2002:SA:646334.687813}。BFT-CUP~\cite{Alchieri:2008:BCU:1496310.1496316}容许未知参与者,但仍然预设一个避免女巫的中心化的准入控制机制。

一般说来,拜占庭协商系统中的成员资格通过中心权威或者封闭式协商。之前去中心化准入尝试已经放弃一些优势。Ripple采用的方式,是发布一个参与者他们自己可以修改的``启动器''成员列表,并希望人们的修改微不足道或者被绝大多数参与者所重现。不幸的是因为不同的列表导致安全担保无效~\cite{ripple2014},实际情形下用户不愿意编辑这个列表,于是巨大的权力最终集中到启动器列表的维护者。另一种方式为Tendermint所采用~\cite{tendermint2014},它基于权益证明的成员资格。然而这样做又把信任捆绑到资源所有权上了。SCP是第一个给参与者最大化的自由来选择信任哪些其它参与者组合的拜占庭协商协议。
\section{联邦拜占庭一致性系统}

本节介绍联邦拜占庭一致性(FBA)模型。和非联邦一致性一样,FBA需要解决更新重复状态的问题,例如事务分类帐或认证树。通过认可采用什么样的更新,节点避免了冲突的、不可协调的状态。我们通过一个唯一的槽(slot)来识别更新,这使得我们可以推断出更新依赖。例如,在一个顺序记录的日志中槽可以是连续标号的位置。

FBA系统运行一个可以确保节点认可槽的内容的共识协议。节点$v$在它已经安全地更新了槽$i$所依赖的所有槽之后可以在槽$i$处安全地更新$x$;另外,它相信所有正确工作的节点最终将会认可槽$i$处的更新$x$。外部系统会以不可逆的方式对具体的值做出反应,因此一个节点不能够在之后改变这些值。

FBA系统的一个挑战是恶意的团体可能参与多次并在数量上超过诚实节点。因此,传统的基于大多数数量的quorum机制无法适用。FBA采用一种分布式的方式来决定quorum,这是通过每个节点选择``quorum切片''来实现的。接下来的子章节讨论了一些例子。最后,我们定义了一个共识协议应当希望达成安全性和完整性的关键属性。

\subsection{{\quorum}切片}

在一个共识协议中,节点交换消息来对关于{\slot}的陈述进行断言。我们假设这些断言无法被伪造——如果节点通过公钥命名并且数字签名了这些消息那么这是可以保证的。当一个节点侦听到足够多的节点断言了某个陈述之后,它将假定不会有工作节点否定这一陈述。我们称这样的一个足够的集合为\textbf{{\quorum}切片},或简称\textbf{切片}。为了允许在节点失败的情形下系统仍然能够推进,一个节点可能有多个切片,它的任意一个切片都足够让它相信某个陈述。从较高层次来看,一个FBA系统由松散的节点邦联组成,其中的每个节点包含一个或多个切片。更形式化地讲:

\begin{definition}[FBAS]
	一个联邦拜占庭协商系统,或简称\textit{FBAS},是一个二元组$\langle\mybm{V}, \mybm{Q}\rangle$,它包含节点集合$\mybm{V}$和{\quorum}函数$\mybm{Q}:\mybm{V}\rightarrow2^{2^{\mybm{V}}}\backslash\left\{\emptyset\right\}$,后者用于指定每个节点的一个或多个切片,这里一个节点属于所有它自己的切片——即,$\forall v\in\mybm{Q},\forall q\in{\mybm{Q}(v)},v\in q$(注意$2^X$指的是$X$的幂集)。
\end{definition}

\begin{definition}[\quorum]
	FBAS$\langle\mybm{V},\mybm{Q}\rangle$中的节点集合$U\subseteq\mybm{V}$是一个\textbf{\quorum}当且仅当$U\neq \emptyset$且$U$包含每个节点的一个切片——即,$\forall v\in U, \exists q \in \mybm{Q}(v), s.t. q\subseteq U$。
\end{definition}

{\quorum}是足够达成一致的节点集合。切片是说服某一特定节点认可的{\quorum}子集。一个{\quorum}切片可能小于{\quorum}。考虑\todo{图2}中的四节点系统,每个节点包含单一切片而箭头指向切片中的其他成员。节点$v_1$的切片$\left\{v_1,v_2,v_3\right\}$足以说服$v_1$认可。但是$v_2$和$v_3$的切片包含$v_4$,这意味着$v_2$和$v_3$都不能在没有$v_4$同意的情况下断言某个陈述。因而,没有$v_4$的参与就不可能达成一致,而唯一的包含$v_1$的{\quorum}是所有节点的集合$\left\{v_1,v_2,v_3,v_4\right\}$。

传统非联邦共识要求所有的节点接受相同的切片,即$\forall v_1, v_2, \mybm{Q}(v_1)=\mybm{Q}(v_2)$。因为任意的切片都足够使得所有的节点相信所有节点的某一个陈述,中心化系统不加区分切片和{\quorum}。缺点在于成员关系和{\quorum}必须事先规定好,妨碍了开放式成员关系和去中心化控制。传统的系统,例如PBFT~\cite{Castro:1999:PBFT},通常有$3f+1$个节点,它们中的任意$2f+1$个节点组成一个{\quorum}。这里$f$是拜占庭错误的最大值——意味着节点的随意行为都能保证系统能够存活。

本文中介绍的FBA一般化了中心化共识以使得它可以容纳更广泛的环境。FBA的关键不同在于每个节点选择它自己的{\quorum}切片集合$\mybm{Q}(v)$。从而系统范围内的{\quorum}由每个节点的独立决策产生。节点可以根据任意准则选择切片,例如节点的信誉或财务安排。在一些环境下,可能对任何节点来说记录整个系统的所有节点集合${\mybm{V}}$是不现实的,然而仍然可以达成共识。
\subsection{例子和讨论}
\todo{图3}展示了一个层状系统,系统中的不同节点有着截然不同的切片集合,这可能只有在FBA才能实现。顶层由$v_1,\ldots,v_4$组成,其结构类似于$PBFT$中$f=1$的情形,这意味着只要其他三个节点可达并正常工作,它可以容忍一个拜占庭故障。节点$v_5,\ldots,v_8$组成中间层并且不相互依赖,而是依赖于顶层。中间层的节点只要求两个顶层节点就可以形成切片。(假定顶层只有最多一个拜占庭故障,那么除非整个系统出错否则两个顶层节点不会同时出现故障。)节点$v_9$和$v_10$在叶子层,其切片由任意两个中间层节点组成。注意这里$v_9$和$v_10$可能选择不相交的切片集合,例如$\left\{v_5,v_6\right\}$和$\left\{v_7,v_8\right\}$;然而,两者都会间接依赖于顶层节点。

实际情形中,顶层可能包含来自各个地方的从四到几十个广为人知并可信的金融机构。当顶层的大小增长时,可能不再有关于它的成员关系的准确认可,但是将会有顶层成员间意识上很大程度的重叠。另外,我们可以想象很多的中间层,例如每个代表一个国家或者地理区域。

这种分层系统和域名间的网络路由系统十分类似。当今的网络是由独立的对等直连和网络对间的传输关系共同组成的。没有中央权威机构来指派或仲裁这些安排。然而这些成对的关系也已足够创建出实际意义上的第一层结构——网络服务提供商(ISP)~\cite{peer_isp2010}。	尽管英特网的可达性受防火墙影响,但传递性的可达性几乎是完好的---例如,某个防火墙可能会阻塞纽约时报,但如果它允许Google访问,而Google能够访问纽约时报,那么纽约时报也间传递性地可达。传递性可达对网站来说或许是受限的设施,但是这对共识至关重要;等价的例子是Google仅当在纽约时报接受某个陈述的时候才接受该陈述。

如果我们把{\quorum}切片看成类似网络可达性,而把{\quorum}看成是传输性可达,那么网络的几乎完全的传输性可达暗示我们同样也可以利用FBA达成世界范围内的共识。在很多方面,共识比网际间的路由转换要容易得多。传输消耗资源并且花费资金,但包含切片的过程仅仅要求检查数字签名。因此,FBA节点可以在包含切片的那端报告错误,相比较常见的在对等直连和传输安排见到的那样,这可以用更为相互依赖且冗余的方式来建立保守的切片。

另一个对中心化网络来说不可能实现完成的是有环依赖结构,例如\todo{图4}中展示的那样。这样一个环状结构不会被有意生成出来,然而当独立的嗯节点选择他们自己的切片时,这可能导致整个网络最终被植入了环状依赖。更大的问题是,相比较传统的拜占庭一致性来说,一个FBA的协议必须解决远为多样的{\quorum}结构。
\subsection{安全性和存活性}\label{sec:fba-safe-live}

我们把节点分为\textit{良性行为的}和\textit{恶性行为的}。一个良性行为的节点选择可察觉的{\quorum}切片(将在第\ref{sec:resilience}节深入讨论)并且遵守规则,这包括最终回应所有的请求。一个恶性行为的节点不是这样。恶性行为的节点受拜占庭故障的影响,这意味着它们可能会有随意的行为。例如,一个恶性节点可能会被入侵而它的所有者可能恶意地修改软件,或者它也可能崩溃。

拜占庭一致性的目标是要确保良性行为的节点要在恶性行为节点存在的情形下对外界要展示相同的值。这个目标包含两个方面。首先,我们希望防止节点分叉并对同意槽向外界显示不同的值。其次,我们需要确保节点的确会向得到值,而不是在某个死的状态下阻塞了,这种情形下的共识变得不再可能。我们引入下面为这些属性引入下面两个术语:

\begin{definition}[安全性]
	FBAS中的一组节点集合,如果当中的任意两个节点都不会向外界展现出不同的值,那它被认为是{\textbf{安全的}}。
\end{definition}

\begin{definition}[存活性]
	在FBAS中,如果给予合适的消息发送和时间限制,当节点能够向外界产生具体的值的话,那么该节点被认为是{\textbf{存活的}}。
\end{definition}

我们称这些既是\textit{安全的}又是\textit{存活的}节点是\textit{正确的}。不正确的节点被称为{\textit{有故障的}}。一个恶性行为的节点是有故障的,但一个良性行为的节点也可能是有故障的---在它无限等待一个恶性行为的节点的消息时,或者更为严重的是它的状态被恶性行为的节点用错误的信息给破坏了。

\todo{图5}强调了这样一种可能的节点故障。左边是拜占庭故障---恶性行为的节点。右边是两类良性行为的但有故障的节点。不具有存活性的节点被标成了\textit{阻塞的},而那些不具有安全性的节点被标成了\textit{分叉的}。一种破坏安全性的攻击严格地比破坏存活性的攻击更为强大。因此我们把分叉节点归为阻塞节点的子集。

我们对\textit{存活性}的定义是比较弱的,这是因为它承认\textit{持久性优先抢占},在这样一种状态下持久性地有可能达成共识,但是网络网络持续地用错误的方式延迟或调整重要信息的顺序来阻碍共识的形成。持久性优先抢占在一个纯异步、确定的、并能够免受节点故障干扰的系统里面是不可避免的~\cite{Fischer:1985:IDC:3149.214121}。幸运的是,抢占是暂时的。它并不意味着节点故障,因为系统可能会在任何时刻恢复。协议可以通过随机性(接入节点总是选择随机性的候选值直到有足够多的正好选择了同一个节点~\cite{Ben-Or:1983:AFC:800221.806707,Bracha:1985:ACB:4221.214134})或对消息延迟的现实情况下的假设~\cite{Dwork:1988:CPP:42282.42283}来缓解这个问题。当希望限制执行时间时后一种解决方案更加实际。当然,只有终止性要求而不是安全性要求依赖于消息超时设置。



\section{最优恢复}\label{sec:resilience}

节点是否安全且可活依赖于多方面因素:它们选择了什么样的{\quorum}切片,哪些节点是恶性行为的,当然还有具体的共识协议和网络行为。像对待通常的异步系统一样,我们假设系统最终将在良性行为的节点之间传递消息,但是可能会随意性地延迟或重新排序消息。

本节回答下面这个问题:给定一个特定的$\langle\mybm{Q},\mybm{V}\rangle$和恶性行为节点子集$\mybm{V}$,在忽略网络因素的情形下任何联邦拜占庭一致性协议能够确保的最佳安全性和存活性是什么?我们首先讨论{\quorum}交这一属性,没有它安全性是无法保证的。接着我们引入可去集的概念——这是一个错误节点的集合,不考虑它们系统仍然有可能保持安全性和存活性。

\subsection{{\quorum}交}\label{sec:quorum_intersect}

仅当用函数$\mybm{Q}$表示的{\quorum}切片满足一种有效性属性的时候协议才能保证可达到协商,我们把这一属性称为{\quorum}交。

\begin{definition}[{\quorum}交]
	一个FBAS包含\textbf{{\quorum}交}当且仅当它的任意两个{\quorum}共享一个节点,即$\forall U_1\in\mybm{Q},U_2\in\mybm{Q},U_1\cap U_2\neq\emptyset$。
\end{definition}

\todo{图6}展示了一个不具有此特性的系统,这里$\mybm{Q}$允许两个{\quorum}$\left\{v_1,v_2,v_3\right\}$和$\left\{v_4,v_5,v_6\right\}$,它们彼此不相交。不相交的集合可能会独立地认可相互矛盾的陈述,暗中破坏系统范围内的一致性。当很多的{\quorum}存在时,如果存在两个不相交的话那么{\quorum}交将会失败。例如,\todo{图6}中的节点集合$\left\{v_1,\ldots ,v_6\right\}$是一个和另外两个{\quorum}相交的{\quorum},但是系统本身仍然缺少{\quorum}交。

没有一种协议能够在没有{\quorum}交的情形下保证安全性,因为这样的配置可能会以两个对对方一无所知的FBAS系统的方式运行着。然而即使有{\quorum}交,在存在恶性行为节点的情形下安全性也可能无法保证。比较有两个不相交集合的\todo{图6}和在恶性行为的节点$V_7$处有交的两个{\quorum}的\todo{图7}。如果$v_7$产生对左右两个{\quorum}产生两个不相容的陈述,其结果等价于不相交{\quorum}。

事实上,由于恶性节点对安全性毫无用处,任何一个协议必须在良性行为节点在它们自己那里享有{\quorum}交的情形下才能保证安全性。毕竟,在安全性最坏的场景下,所有的恶性行为的节点都可能持续性地做出不一致的陈述来使得{\quorum}产生分歧。两个仅在恶性节点处相重叠的{\quorum}由于恶性节点的欺骗同样会像运行在两个不同的FBAS系统里一样。总之,FBAS$\langle \mybm{V},\mybm{Q}\rangle$可以在节点集合$B\subseteq \mybm{V}$的拜占庭错误中生存下来当且仅当$\langle\mybm{Q},\mybm{V}\rangle$在从$\mybm{V}$和$\mybm{Q}$中的所有切片中删除了$B$中的节点之后仍然有{\quorum}交。更形式化的描述是:

\begin{definition}[删除]
	如果$\langle\mybm{Q},\mybm{V}\rangle$是一个FBAS并且$B\subseteq \mybm{V}$是一个节点集合,那么从$\langle\mybm{V},\mybm{Q}\rangle$中\textit{删除}$B$(记作$\langle\mybm{V},\mybm{Q}\rangle^B$),意味着计算修改过后的FBAS $\langle\mybm{V}\backslash B, \mybm{Q}^B\rangle$,这里$\mybm{Q}^B(v)=\left\{q\backslash B|q\in\mybm{Q(v)}\right\}$。
\end{definition}

每个节点$v$有责任确保$\mybm{Q}(v)$不会违背{\quorum}交。一种做法是选择保守的切片,这会导致较大的{\quorum}。当然,一个恶意的$v$可能会故意地选择能够违反{\quorum}交的$\mybm{Q}$。然而一个恶意的$v$还可能对$\mybm{Q}(v)$的值撒谎或者忽略$\mybm{Q}(v)$来做出随意的断言。总之,当$v$是恶性行为的时候$\mybm{Q}(v)$的值是没有意义的。这就是为什么安全性的必要属性——在删除恶性行为节点之后的良性行为节点仍然有{\quorum}交——是不受恶性行为节点的切片影响的。

假定\todo{图6}是由一个三节点$v_1$, $v_2$, $v_3$含{\quorum}交的FBAS系统进化成的一个六节点无{\quorum}交的FBAS系统。当$v_4$, $v_5$, $v_6$加入时,他们恶意地选择切片来破坏{\quorum}交,这样没有节点能够为$\mybm{V}$确保安全性。幸运的是,删除这些节点得到$\langle\mybm{Q},\mybm{V}\rangle^{\left\{v_4,v_5,v_6\right\}}$重新恢复{\quorum}交属性,这意味着至少$\left\{v_1,v_2,v_3\right\}$能够保证安全性。注意这里的删除是概念性的,用于表述最优安全性。一个协议应当为$v_1$, $v_2$, $v_3$保证安全性而不需要它们意识到$v_4$, $v_5$, $v_6$是恶性行为的。
\subsection{可去集($DSet$)}
我们用可去集(或$DSet$)获得节点切片选择的的容错性。通俗地讲,在一个$DSet$之外的安全性和可活性是可以保证的,	而不需要考虑$DSet$中的节点的行为。换句话说,在一个最优的可恢复的FBAS中,如果一个$DSet$包含每个恶性行为节点,它也会包含每个故障节点;而反之所有在$DSet$之外的节点都是正确的。举例来说,在一个有$3f+1$个节点且{\quorum}大小为$2f+1$的中心化FBPT系统中,任意小于等于$f$个节点可以组成一个$DSet$。由于FBPT事实上能够在有$f$个故障节点的情形下存活下来,它的健壮性是最优的。

在不那么典型的\todo{图3}的例子中,$\left\{v_1\right\}$是一个$DSet$,这是因为顶层的节点可能会出故障而不影响系统的其余部分。$\left\{v_9\right\}$也是一个$DSet$,这是因为其他的节点的正确性都不依赖$v_9$。$v_6,\ldots,v_10$是一个$DSet$,这是因为$v_5$和顶层节点都不会依赖于任意其他5个节点。$\left\{v_5,v_6\right\}$\textit{不是}一个$DSet$,因为它是$v_9$和$v_10$的一个切片,因此如果他们是完全恶意的话,它可能对$v_9$和$v_10$撒谎,并且让它们断言相互矛盾或和系统内其他节点矛盾的陈述。

为了防止一个错误行为的$DSet$影响其他节点的正确性,两个性质必须满足。对于安全性,删除$DSet$不会暗中破坏{\quorum}交。对于可活性,$DSet$不能够否认其他节点的工作{\quorum}。这使得我们有以下定义:

\begin{definition}[DSet]
	令$\langle\mybm{V},\mybm{Q}\rangle$是一个FBAS而$B\subseteq \mybm{Q}$是一个节点集合,我们称集合$B$是可去集当且仅当:
	\begin{enumerate}
		\item (除$B${\quorum}可交性) $\left\{\mybm{V},\mybm{Q}\right\}^{B}$是{\quorum}可交的;
		\item (除$B${\quorum}可达性) 要么$\mybm{Q}\backslash B$是$\langle\mybm{V},\mybm{Q}\rangle$中的一个{\quorum},要么$B=\mybm{V}$。
	\end{enumerate}
\end{definition}

除$B${\quorum}可交性防止$B$中的节点拒绝回复请求或阻塞其他节点运行。除$B${\quorum}可达性防止对立的情形---$B$中的节点制造矛盾的断言而让其他的节点对同一个槽对外产生不一致的值。在切片选择时节点必须权衡这两类威胁。其他条件相同的情况下,大一点的切片会产生大一点的{\quorum}从而会有更多的重叠,这意味着更少的故障节点集合$B$在删除时将会破坏{\quorum}交。另一方面,大一点的切片更可能包含故障节点,这会危及到{\quorum}的可达性。

最小的包含所有恶性行为的节点的$DSet$可能也会包含良性行为的节点;这反映了这样一个事实:一个足够大的恶性行为的节点集合可能会导致良性行为的节点出现故障。例如,在\todo{图3}中,包含$v_5$和$v_6$的最小$DSet$是$\left\{\v_5,v_6,v_9,v_10\right\}$。在一种特殊的情形下,$\mybm{V}$是$DSet$。这个特殊情形的动机在于,如果所有的节点都出现了故障,那么剩余的(零个)节点自然是正确的。给定足够多的恶性行为的节点,包含所有节点的$\emph{V}$可能是包含所有恶性行为节点的最小$DSet$,这意味着没有协议能够保证有比整个系统失败更好的结果了。

一个FBAS系统中的$DSet$是由{\quorum}函数$\mybm{Q}$事先决定的。而哪个节点的行为是良性或是恶性的有运行时行为决定,例如机器被入侵了。我们关心的$DSet$是那些包含恶性行为的节点,因为它们能够帮助我们把可以确保正确的节点从不能保证正确的节点中辨别出来。有鉴于此,我们引入下面的术语:

\begin{definition}[完整的]
	FBAS中的节点$v$是\textbf{完整的}当且仅当存在一个包含所有恶性行为的$DSet$集合$B$且$v\not\in B$。
\end{definition}

\begin{definition}[被污染的]
	FBAS中的节点$v$是\textbf{被污染的}当且仅当它不是完整的。
\end{definition}

即使一个节点$v$本身是良性行为的,当它被足够多的故障节点所包围而被阻塞进程或状态被破坏时,它也是被污染的。FBAS都不能保证被污染节点的正确性。\todo{图8}概述了节点的关键属性。下面的定理减轻了分析的困难,它们表明被污染的节点集合总是FBAS中的一个有{\quorum}交的$DSet$。

\begin{theorem}\label{th1}
	令$U$是FBAS$\langle\mybm{V},\mybm{Q}\rangle$中的一个{\quorum},$B\subseteq \mybm{Q}$是节点集合,并令$U^\prime=U\backslash B$。若$U^\prime\neq \emptyset$,则$U^\prime$是$\langle\mybm{V},\mybm{Q}\rangle^B$中的一个{\quorum}。
\end{theorem}

\begin{proof}
	因为$U$是一个{\quorum},每个节点$v\in U$都有一个$q$使得$q\in U$。由于$U^\prime\subseteq U$,则有对每个节点$v\in U^\prime$存在$q\in\mybm{Q}(v)$使得$q\backslash B\subseteq U^\prime$。重新使用删除记号书写即得$\forall v\in U^\prime,\exists q\in\mybm{Q}^B(v)$使得$q\subseteq U^\prime$;因为$U^\prime \subseteq \mybm{Q}\backslash B$,这表示$U^\prime$是$\langle\mybm{V},\mybm{Q}\rangle^B$的一个{\quorum}。
\end{proof}

\begin{theorem}\label{th2}
	如果$B_1$和$B_2$是FBAS$\langle\mybm{V},\mybm{Q}\rangle$的一个有{\quorum}交的$DSet$,那么$B=B_1\cap B_2$也是一个$DSet$。
\end{theorem}

\begin{proof}
	令$U_1=\mybm{Q}\backslash B_1$且$U_2=\mybm{Q}\backslash B_2$。如果$U_1=\emptyset$,那么$B_1=\mybm{V}$且$B=B_2$,是一个$DSet$。类似地,如果$U_2=\emptyset$,那么$B=B_1$,这种情况已解决。否则,注意到除$DSet$的$B_1$和$B_2$的{\quorum}可达性,$U_1$和$U_2$是$\langle\mybm{V},\mybm{Q}\rangle$。由定义可知,两个{\quorum}的并仍然是{\quorum}。因此${\mybm{Q}\backslash B}=U_1\cup U_2$是一个{\quorum}且它有除$B${\quorum}可达性。
	
	我们需要说明除$B${\quorum}可交性。令$U_a$和$U_b$是${\mybm{V},\mybm{V}^{B}}$的任意两个{\quorum},$U=U_1\cap U_2 = U_2\backslash B_1$。根据$\langle\mybm{V},\mybm{Q}\rangle$的定义,$U=U_1\cap U_2  \neq \emptyset$。但由定理\ref{th1},$U=U_2\backslash B_1$必为$\langle\mybm{V},\mybm{Q}\rangle^{B_1}$。考虑到$U_a\backslash B_1$和$U_a\backslash B_2$不可能同时为空集,否则$U_a\backslash B$为空与$U_a$是{\quorum}的定义矛盾。因此,由定理\ref{th1},要么$U_a\backslash B_1$是$(\langle\mybm{V},\mybm{Q}\rangle^{B})^{B_1}=\langle\mybm{V},\mybm{Q}\rangle^{B_1}$,要么$U_a\backslash B_2$是$\langle\mybm{V},\mybm{Q}\rangle^{B_2}$,或者两者都是。在前一种情形下,注意到假如$U_a\backslash B_1$是$\langle\mybm{V},\mybm{Q}\rangle^{B_1}$的{\quorum},那么由$\langle\mybm{V},\mybm{Q}\rangle^{B_1}${\quorum}可交性可得$(U_a\backslash B_1)\cap U\neq \emptyset$;由于$(U_a\backslash B_1)\cap U = (U_a\backslash B_1)\backslash B_2$,可知$U_a\backslash B_2\neq \emptyset$,从而$U_a\backslash B_2$是$\langle\mybm{V},\mybm{Q}\rangle^{B_2}$中的一个{\quorum}。类似地,$U_b\backslash B_2$也必是$\langle\mybm{V},\mybm{Q}\rangle^{B_2}$。但除$B${\quorum}可交性告诉我们$(U_a\backslash B_2)\cap (U_b\backslash B_2)\neq \emptyset$,这仅在$U_a\cap U_b\neq \emptyset$的时候可能。
\end{proof}

\begin{theorem}
	在一个有{\quorum}交的FBAS中,被污染的集合是一个$DSet$。
\end{theorem}

\begin{proof}
	令$B_{min}$是一个包含所有恶性行为节点的$DSet$的交。由\textbf{完整性}的定义可知一个节点$v$是完整的当且仅当$v\not\in B_{min}$。因此$B_{min}$就是被污染节点集合。由定理\ref{th2}可知,$DSet$在交的意义下是闭的,因此$B_{min}$也是一个$DSet$。
\end{proof}

\section{联邦选举系统}

本节开发了一套可供FBAS中的节点对某个陈述认可的联邦投票技术。从高层次来看,对某个陈述$a$认可的过程涉及节点交换两个集合的消息。首先,节点为$a$投票。其次,如果投票成功,节点\textit{确认}$a$,在第一次投票成功的基础上有效地举行第二次投票。

在每个节点看来,两个回合的消息把对陈述$a$的认可分成了三个阶段:不知道的,认可的和确认的。至少,$a$的状态对一个节点来说是一无所知的---$a$可能最终为真,为假,甚或\textit{卡在}了一个长期不确定的状态上。如果第一次投票成功,$v$可能会\textit{接受}$a$。两个完整的节点从来不会接受相互矛盾的成熟,因此如果$v$是完整的并且接受了$a$那么$a$就不可能是假的。

然而由于两种原因,$v$接受$a$并不能说明$a$是真的。首先,$v$接受$a$这一事实并不代表所有完整的节点都能接受。其次,如果$v$是被污染的,那么接受$a$不能说明任何问题---$a$可能在完整节点处被认为是假的。然而即使$v$被污染了---$v$并不知道---系统仍然可以有良性行为节点的{\quorum}可交性,在这种情形下为了最优安全性,$v$需要对$a$的断言的更高的保证。举行第二次选举用来解决这两个问题。如果第二次选举成功,$v$转向到\textit{已确认}状态,这时它可以最终认为$a$是真并且作用于$a$上。

下面子章节阐述了联邦选举过程的细节。因为选举并没有去除被阻塞的陈述的可能性,第\ref{sec:vote_stuck}节讨论如何处理它们。在第\ref{sec:scp}节里我们将联邦选举传统转化成一个共识协议,它可以避免完整节点的被阻塞槽的可能性。

\subsection{开放式成员关系下的投票系统}
\subsection{阻塞集}
\subsection{接受陈述}
\subsection{可接受性是不够的}

\subsubsection{安全性}
\subsubsection{可活性}
\subsubsection{和中心化投票的比较}
\subsection{陈述确认}

两类被接受陈述的局限性都来源于这一事实:一个完好节点$v$可能会给一个已经被批准的陈述$a$投反对票。在反对$a$之后,它不会再投赞成票,这使得$v$将不可能批准$a$。为了给$v$提供一种在给$a$投反对票之后仍然能够批准$a$的方法,\textit{接受}的定义给出了一个基于{\vblock}的第二准则。但是这个第二准则弱于批准,对有{\quorum}交的被污染节点没有任何保障。

现在如果一个陈述$a$有``从来没有任何完好节点反对它''这样一个属性,那么我们没有必要去接受它。完好节点可以简单地批准$a$,而我们可以在作用于$a$之前要求它们这样做。我们成这样的陈述是不可驳斥的。

\begin{definition}[不可驳斥]
	当没有完好节点可以反对一个陈述$a$时,称这条陈述在FBAS中是\textbf{不可驳斥的}。
\end{definition}

定理\ref{thm:intact_cannot_accept_contradictory}告诉我们两个完好节点不可能接受相互冲突的陈述。因此,尽管一些节点可能会反对某个被完好节点接受的陈述$a$,对``一个完好节点接受了$a$''这一陈述是不可反驳的。这建议我们举行第二次投票来批准``一个完好节点接受了$a$''这一事实。

\begin{definition}[确认]
	在FBAS中一个{\quorum}$U_a$\textbf{确认}一个陈述$a$当且仅当$\forall v\in U_a$, $v$声称承认$a$。一个节点\textbf{确认}$a$当且仅当它在这样一个{\quorum}中。
\end{definition}

节点通过声称``$accept(a)$''的接受陈述$a$,这是``一个完好节点接受$a$''的缩略形式的陈述。一个良性行为的节点$v$仅当在接受了$a$之后才能投票赞成$accept(a)$,这是因为$v$不能假定其它的任何节点都是完好的。如果$v$本身是被污染的,$accept(a)$可能会失败,在这种情形投票赞成它可能会牺牲$v$的存活性,然而无论如何一个被污染的节点对存活性是没有任何保证的。

下一个定理表明节点能够依赖被确认陈述而不损失最佳安全性。定理\ref{thm:confirmed_stats_keep_liveness}则说明被确认陈述符合第\ref{sec:accept_not_enough_liveness}节中\textit{认可}的定义,这意味着节点可以依赖被确认陈述而不会破坏完好节点的存活性。

\begin{theorem}\label{thm:confirmed_stats_keep_safety}
	设$\langle\mybm{V},\mybm{Q}\rangle$是一个满足除$B${\quorum}可交性的FBAS,且假定$B$包含了所有的恶性行为节点。令$v_1$和$v_2$是不在$B$中的两个节点。设$a$和$\bar a$是相互冲突的陈述。则有,若$v_1$确认了$a$,则$v_2$不会确认$\bar a$。
\end{theorem}

\begin{proof}
	首先注意到$accept(a)$和$accept(\bar a)$相互冲突——没有一个良性行为的节点会同时投票赞成两者。更进一步,注意到$v_1$必须批准$accept(a)$以确认$a$。由定理\ref{thm:nodes_cannot_ratify_contracdictory},$v_2$不会批准$accept(\bar a)$因此不会确认$\bar a$。
\end{proof}

\begin{theorem}\label{thm:confirmed_stats_keep_liveness}
	如果一个含{\quorum}交的FBAS $\langle\mybm{V},\mybm{Q}\rangle$确认了一个陈述$a$,那么不论接下来会发生什么,一旦足够的消息被发送并处理了,每个完好节点将会接受并确认$a$。
\end{theorem}

\begin{proof}
	令$B$是被污染节点的$DSet$,并令$U_a\not \subseteq B$是一个{\quorum},一个完好节点通过它确认$a$。让在$U_a\backslash B$中的节点广播$accept(a)$。根据定义,任何节点$v$,不论它是如何投票的,在从一个{\vblock}集合中接受了$accept(a)$陈述之后将接受$a$。因此,这些消息会说服额外的节点去接受$a$。让这些额外节点反过来广播$accept(a)$直到到达这样一个时刻:不论进一步的通信如何,没有更多的完好节点会接受$a$。一旦这个过程完成,设$V^+$是已经接受$a$的所有完好节点集合,设$V^-=(\mybm{V}\backslash B)\backslash V^+$是那些不能接受$a$的完好节点集。为证明所有的完好节点都接受了$a$,我们必须证明$V^-=\emptyset$。

	由定理\ref{thm:quorum_subset_is_quorum},$U_a\backslash B$是$\langle\mybm{V},\mybm{Q}\rangle^{B}$中的一个{\quorum}。由于对于任何$v\in V^-$来说$V^+$都不是{\vblock}的,那么由定理\ref{thm:quorum_availability_vs_vblocking} 要么$V^-=\emptyset$要么$V^-$是$\langle\mybm{V},\mybm{Q}\rangle^{B}$中的一个{\quorum}。后一种情形导致这样一个矛盾:由于$V^-$只包含完好的(因此也是良性行为的)节点,它们当中的没有能够在首先真正接受$a$的情况下声明$accept(a)$,这意味着$U_a\backslash B\cap V^-=\emptyset$。然而这是不可能的,因为除$B$(一个$DSet$){\quorum}可交性告诉我们它的反面:$U_a\backslash B\cap V^-\neq\emptyset$。剩下只可能是$V^-=\emptyset$。一旦每个节点都接受了$a$,所有的都会投票赞成确认$a$。因为完好节点构成了一个{\quorum},这些投票将会成功。
\end{proof}


\begin{figure}
\centering
\begin{tikzpicture}[thick,
    node/.style={draw,ellipse,inner xsep=-1.5mm,align=center,text
      width=2.2cm,
      inner ysep=0pt,
      text depth=1ex,text height=1em,fewshade=\maincolor},
    transition/.style={->,very thick, align=center,shorten
      >=0pt,shorten <=-3pt,font=\small\openup-.5\jot,draw=few-gray},
  ]
\node[node] (N) at (0,0) {未提交};
\node[node] (P) at (3,1) {已赞成~$a$};
\node[node] (A) at (6,1) {已接受~$a$};
\node[node] (C) at (9,1) {已确认~$a$};
\node[node] (nP) at (3,-1) {已赞成~$\na$};
\begin{scope}[on background layer]
\draw[transition] (P.60) to[out=60,in=180]
    (4.5,2) 
    to[out=0,in=120,pos=0]
    node[above,text depth=.5pt,align=left] {\quorum
      赞成\\或接受$a$}
    (A.120);
\draw[transition] (A.60) to[out=60,in=180]
    (7.5,2) to[out=0,in=120]
    node[above,pos=.01,text depth=0pt,align=left]
    {\quorum 确认 $a$}  (C.120);
\draw[transition] (N) to[out=90,in=180,looseness=.9]
     node[above,align=center] {$a$是有效的} (P);
\draw[transition] (N) to[out=-90,in=180,looseness=.9] (nP);
\draw[transition,-] (P.-60) to[out=-60,in=180] (4.5,0);
\draw[transition] 
       (nP.60) to[out=60,in=180] (4.5,0)
       to[out=0,in=-120]
       node[below right,inner sep=0,pos=.1,align=left]
            {{\vblock}集\\[1pt] 接受 $a$}
       (A.-120);
\end{scope}
\end{tikzpicture}
\caption{一个被接受的陈述$a$在节点$v$处可能的状态}
\label{fig:node-states}
\end{figure}

{图\ref{fig:node-states}}总结了一个完好节点为了确认$a$可以采用的路径。在不知情的状态下,$v$可能会赞成$a$或者与之冲突的$\bar a$。如果$v$投票赞成$\bar a$,它之后不会再赞成$a$;但是如果一个{\vblock}集合接受了$a$那么$v$却可以接受$a$。接下来确认消息的{\quorum}允许$v$去确认$a$,由定理\ref{thm:confirmed_stats_keep_liveness}这意味着系统认可$a$。
\subsection{可活性和中立性}
\section{SCP:一种联邦拜占庭一致性协议}\label{sec:scp}

\subsection{提名协议}

\subsubsection{具体的提名协议}
\subsection{表决协议}
一旦节点有了合成值它们将参与表决协议,尽管提名会继续更新合成值。一个表决$b$一个形如$b=\langle n,x\rangle$的二元组,这里$x\neq \perp$是一个值,而$b$是对讨论中的槽具体化的请示书(referendum)。$n\geq 1$是一个确保大些的表决数总是可访问的计数器。我们使用类C语言的标记$b.n$和$b.x$来表示表决$b$的计数和值的域,因而有$b=\langle b.n, b.x\rangle$。表决是全序的,而$b.n$比$b.x$更为重要{\footnote{译注:指二元组$b_1\prec b_2$当且仅当$b_1.n < b_2.n$或$b_1.n=b_2.n $且$b_1.x < b_2.x$;$b_1\equiv b_2$当且仅当$b_1.n=b_2.n\cap b_1.x=b_2.x$。}}为了方便起见,一个特殊无效的空表决$\mybm{0}=\langle 0,\perp\rangle$小于其他任何表决,而一个特别的计数器$\infty$大于其他所有的计数器。

我们分别用提交或终止一个表决$b$作为使用联邦投票来对语句$commib\;b$和$abort\;b$进行认可。对于给定的表决,$commit$和$abort$是相互冲突的,因此一个良性行为的节点最多为它们中的一个投赞成票。在第\ref{sec:voting}节的标注系统下,$commit\;b$的反是$\overline{commit\;b}$,但这里使用$abort\;b$更为直观。

由于对某个槽至多只有一个值被选用,所有提交的和被卡住的表决必须包含相同的值。粗略地说,这意味着如果陈述$commit$和更小的非终止表决相冲突的话那么它是无效的。

\begin{definition}[相容的]
        两个表决是\textbf{相容的}(记作$b_1 \sim b_2$)当且仅当$b_1.x=b_2.x$;它们是\textbf{不相容的}(记作$b_1\not\sim b_2$)当且仅当$b_1.x\neq b_2.x$。我们还将$b_1\leq b_2$(或等价地,$b_2\geq b_1$)且$b_1\sim b_2$记作$b_1\lesssim b_2$($b_2\gtrsim b_1$)。类似地,$b_1\lnsim b_2$或$b_2\gnsim b_1$意味着$b_1\leq b_2$(或等价地$b_2\geq \b_1$)且$b_1\not\sim b_2$。
\end{definition}

\begin{definition}[就绪的]
        一个表决$b$是\textbf{就绪的}当且仅当下面集合中的每个陈述都是正确的:$\left\{abort\;b_{old}|b_{old}\lnsim b\right\}$。
\end{definition}

更准确地说,如果$b$被确认是就绪的话那$commit\;b$对投赞成票来说是有效的,节点通过在对应的终止陈述的联邦投票来保障它。全体一致地对这些陈述进行投票是方便的,因此不论我们在哪里写了``$b$就绪''周围的环境将应用于$abort$陈述的整个集合中。特别地,一个节点投票赞成、接受或确认$b$就绪当且仅当它分别投票赞成、接受或确认它们全部\textit{终止}了。

为了提交一个表决并向外界展示它的值$b.x$,SCP节点首先接受并确认$b$已经就绪,然后接受并确认$commit\;b$。在第一个完好节点投票赞成$commit\;b$之前,经由联邦投票的准备步骤确保所有完好节点最终可以确认$b$是就绪的。当一个完好节点$v$接受$commit\;b$时,意味着$b.x$最终将会被选中。然而,正如第\ref{sec:voting_safety}中所讨论的那样,为了防止$v$被污染$v$必须在作用于它之前确认$commit$。

\subsubsection{具体的表决协议}\label{sec:scp_ballot_concrete}

\todo{图16}强调了由每个节点维护的每一{\slot}的状态。一个节点$v$存储了:它当前的表决$b$;两个最近的已经认定就绪的且不相容的表决对$(p,p^{\prime})$;它必须投票\textit{提交}的(或在后续阶段需要确认\textit{提交}的)最小表决$c$(如果存在的话),对此它还没有接着接受到\textit{终止类}陈述;已确认就绪的最高表决$P$;从每个节点($M$)处接受到的最新消息;以及状态$\varphi$。表决$b$,$p$,$p^{\prime}$和$P$在同一个阶段里是不减的。另外,如果$c\neq\mybm{0}$——意味着$v$可能参与了批准$commit\;c$——代码必须确保$c\lesssim P\lesssim b$。这一不变量保证了节点总是可以投票为当前的表决$b$做好准备。

\todo{图17}展示了协议消息。注意$a\;\vee accept(a)$是每个节点需要为一个{\quorum}所断言的,使得它们按照\textit{接受}定义中的第\ref{itm:cond_normal}种方式接受$a$。每个节点通过设置$b\leftarrow \langle 1,combine(Z)\rangle$,$p\leftarrow \mybm{0}$,$p^{\prime}\leftarrow \mybm{0}$,$P\leftarrow \mybm{0}$,$c\leftarrow \mybm{0}$,$M\leftarrow\emptyset$及$\varphi \leftarrow \textsl{PREPARE}$的方式初始化{\slot}的状态。之后节点在同类间重复地交换消息,发送由 $\varphi$表明的任何消息。一旦给$M$添加了一个新近接受的消息,一个节点$v$按照下面的方式添加它的状态:

\begin{enumerate}\label{protocal_case}
	\item 如果$\varphi = \textsl{PREPARE}$且接受的信息让$v$接受新表决是就绪的,更新$p$和$p^{\prime}$。之后,如果$c\neq \mybm{0}$且$p\gnsim P$或$p^{\prime}\gnsim P$,设置$c\leftarrow \mybm{0}$。
	\item 如果$\varphi = \textsl{PREPARE}$且$v$确认新表决是就绪的,增加$P$。之后,如果$c=\mybm{0}$,$P\geq b$,且$p\gnsim P$或$p^{\prime}\gnsim P$都不成立,则设$c\leftarrow P$且$b\leftarrow P$(尽管通常$b=P$已经成立)。
	\item 如果$\varphi = \textsl{PREPARE}$,且$v$接受一个或多个相容表决的\textit{提交类}消息。设$c$为最小的这类表决,$P$设为最大的使得``$v$能够接受所有的$\left\{commit\;b^{\prime}|c\lesssim b^{\prime}\lesssim P\right\}$,$b\leftarrow \langle \infty, c.x\rangle$和$\varphi\leftarrow \textsl{CONFIRM}$''的表决。
	\item 如果$\varphi = \textsl{CONFIRM}$,且接受的消息让$v$接受新表决为就绪的,则提升$p$至最高已被接受为就绪态的、且满足$p\sim c$的表决。
	\item 如果$\varphi = \textsl{CONFIRM}$且$v$接受更多的相容的\textit{提交类}消息,提升$p$至最高的``使得$v$接受所有的$\left\{commit\;b^{\prime}|c\lesssim b^{\prime} \lesssim P\right\}$''的表决。
	\item 如果$\varphi = \textsl{CONFIRM}$且$v$对任意$c^{\prime}$确认$commit\;c^{\prime}$,设置$c$和$P$为最低和最高的这类表决,设$\varphi\leftarrow\textsl{EXTERNALIZE}$,具体化$c.x$并结束。
\end{enumerate}

当$c=\mybm{0}$时,上述协议实施联邦选举来确认$b$已经就绪。一旦$c\neq \mybm{0}$,该协议对$commit\;c$ (实际上是介于$c$和$P$之间的相容的表决)实施联邦选举。对确认阶段来说,一旦一个良性行为的节点$v$接受了$commit\;c$,该节点就不会接受或尝试确认任何满足$c^{\prime}\not\sim c$的$commit\;c^{\prime}$。因此直观上说,一旦一个\textit{提交}被确认了,只要节点具有{\quorum}交属性那么具体化它的值就是安全的。

所有来自一个节点的消息在元组$\langle \varphi,b,p,p^{\prime},P\rangle$的定义之下是全序的,这里$\varphi$是最重要的域而$P$最不重要。所有的\textsl{PREPARE}消息都在\textsl{CONFIRM}消息之前,转而对于给定的{\slot}来说在单独的\textsl{EXTERNALIZE}消息之前。\textsl{PREPARE}信息显式地包含这四个域,而\textsl{CONFIRM}和\textsl{EXTERNALIZE}包含\todo{图17}中所描述的值。这一序关系使得$M$值包含来自每个节点的最新表决而不依赖于时间来排序消息成为可能,这是因为网络环境可能会对消息重新排序。

一些协议的细节需要解释。形如``$abort\;b^{\prime}\vee accpet(abort\;b^{\prime})$''的由\textsl{PREPARE}所蕴含的陈述并没有指明$v$是否赞成或确认$abort\;b^{\prime}$。对于\textit{接受}的定义来说这种区分并不重要。掩盖这种区分使得$v$忘记了旧的它投票提交(因此不能够投票终止)的表决--只要它为这些表决接受一个\textit{终止类}消息的话。

为了确保节点收敛于$P$,$p$和$p^{\prime}$都是必需的,这是因为定理\ref{thm:confirmed_stats_keep_liveness}要求节点重新广播它们已经接受的消息。从\textit{就绪}的定义可知,``一个节点接受为就绪态的、两个不相容的最高表决''蕴含了``所有该节点接受为就绪态的表决''。

在$v$发出\textsl{ENTERNALIZE}消息的时候它实际上已经接受了一个区间内的\textit{提交类}消息$\left\{commit\;b^{\prime}|b^{\prime}\gtrsim c\right\}$。然而,$v$设置$P$来断言``只有它确认提交的表决是可接受的'',而不是在隐式的\textsl{CONFIRM}消息中设置$P.n=\infty$从而对每个$b^{\prime}\gtrsim c$断言$accept(commit\;b^{\prime})$。这样做是足够的,因为一旦一个单独的完好节点确认了$commit\;c$,定理\ref{thm:confirmed_stats_keep_liveness}告诉我们所有的完好节点也将确认它。把关注点集中在已被确认的表决上有额外的好处:\textsl{EXTERNALIZE}消息仅断言$v$已经批准的信息,从而使得$\mybm{Q}(v)$不再相关。这意味着一个独立静态的\textsl{EXTERNALIZE}消息对未来任意远处的想赶上进度的节点来说都是有用的,即使{\quorum}切片与此同时已经改变了很多。

交换表决消息只需要一个RPC。参数是发送者最新的消息而返回值是接受者最新的消息。对于\textsl{NOMINATE},如果$D$或在表决中的值$x$是加密哈希,那么为了取回没有被缓存的哈希原像需要一个单独的RPC。
\subsubsection{表决选择}
\subsection{正确性}\label{sec:scp_correct}

一个节点只有在已经许诺确认所有小编号的表决的\textit{终止}陈述之后才能担保确认$commit\;b$陈述。因为一个良性行为的节点不能够接受(因此也不担保确认)相冲突的陈述,这意味着对于给定的$\mybm{V},\mybm{Q}$,定理\ref{th5}确保一个良性行为节点集合$S$只要享有除$\mybm{V}\backslash S${\quorum}可交性则不会产生相互冲突的值。如果$\mybm{V}$和$\mybm{Q}$只在槽间改变的话那么安全性仍然成立,但如果它们在槽中(mid-slot)改变呢(例如用于应对节点崩溃)?为了分析在重新配置的情形下的安全性,我们保守地对旧的和新的{\quorum}切片集合进行交操作;这反映了这样一个事实:节点可能依据来自不同时期的消息的组合来作出决定。因此很保守地讲,一个节点只有在当前槽用到的每个配置下都是完好的我们才说它是完好节点。但是我们可以放松要求而说:如果一个节点在最近的配置中都是完好的并且在以往的配置中从未接受过来自全部由恶性行为节点的{\vblock}集合发来的消息,则我们称该节点是完好的。

\begin{theorem}\label{th12}
	令$\langle \mybm{V_1},\mybm{Q_1}\rangle,\ldots,\langle \mybm{V_k},\mybm{Q_k}\rangle$是一个FBAS在协商一个单独槽点的时候经历过的配置集合。令$\mybm{V}=\mybm{V_1}\cup \cdots\cup \mybm{V_k}$且$\mybm{Q}(v)=\left\{q|\exists j, v\in\mybm{V_j}\cap q\in\mybm{Q}_j(v)\right\}$。令$B\subseteq\mybm{V}$是一个集合,满足$B$包含所有已经发送了非法消息的恶性行为节点---尽管$\mybm{Q}\backslash B$可能仍然包含崩溃(不响应)的节点。假设$v_1\not\in B$具体化了$x_1$,而$v_2\not\in B$具体化了$x_2$。则如果$\langle\mybm{V},\mybm{Q}\rangle^{B}$有{\quorum}交,那么$x_1=x_2$。
\end{theorem}

\begin{proof}
	为了让$v_1$产生具体的$x_1$,它必然已经和一个伪{\quorum}$U_1\subseteq\mybm{V}$合作批准了$accept(commit(\langle n_1,x_1\rangle))$。我们称伪{\quorum}是因为$U_1$对任意特殊的$j$来讲可能都不是$\langle\mybm{V}_j,\mybm{Q}_j\rangle$的{\quorum},这是由于批准可能已经涉及包含多个配置的消息。然而,为了使得批准成功,$\forall v\in U_1,\exists j, \exists q\in\mybm{Q}_j(v)$使得$q\subseteq U_1$。从$\mybm{Q}$的构造方式可知$q\in\mybm{Q}$。因此$U_1$是$\langle\mybm{V},\mybm{Q}\rangle$。类似地,一个伪{\quorum}必然已经批准了$accept(commit\langle n_2,x_2\rangle)$,且$U_2$一定是$\langle\mybm{V},\mybm{Q}\rangle$的一个{\quorum}。根据$\langle\mybm{V},\mybm{Q}\rangle^{B}$的{\quorum}交,必然存在某个$v\in \mybm{V}\backslash B$使得$v\in U_1\cap U_2$。根据假设,这样的$v\not\in B$不能声明接受不相容的表决。由于$v$确认接受值为$x_1$和$x_2$的表决提交,那么必然有$x_1=x_2$。
\end{proof}

对于节点$v$的存活性,当一个FBAS对一个单独节点经历了一系列的重新配置$\langle \mybm{V_1},\mybm{Q_1}\rangle,\ldots,\langle \mybm{V_k},\mybm{Q_k}\rangle$时我们需要考虑一些东西。首先,定理\ref{th12}的安全性前提条件必须对$v$以及$v$关心的节点成立,因为违反安全性会破坏定理\ref{th10}中所要求的{\quorum}交属性。其次,最新状态下的恶性行为节点集合$\langle \mybm{V_k}, \mybm{Q_k}$必须不能是{\vblock}的,这是因为这会否定一个{\quorum}而阻止它批准陈述。最后,$v$的状态不能够被一个{\vblock}集合错误地声称接受$\langle \mybm{V_1},\mybm{Q_1}\rangle\cdots \langle \mybm{V_{k-1},\mybm{Q_{k-1}}\rangle}$的中一个陈述所污染。总之,我们认为节点$v$在下面的条件成立的情况下是完好的。首先,在恶性行为节点被删除时对当前槽的过去的配置的并必须有{\quorum}交。其次,$v$必须在最新的视图$\langle \mybm{V_k}, \mybm{Q_k}$依据以往的静态配置标准还是完好的。最后,$v$必须从未接受过以往的配置$\langle \mybm{V_1},\mybm{Q_1}\rangle\cdots \langle \mybm{V_{k-1},\mybm{Q_{k-1}}\rangle}$中的全有恶性节点组成的{\vblock}集合中接受过消息。
\section{局限性}

SCP只有在节点选用了足够多的{\quorum}切片的情况下能够保证安全性。	第\ref{sec:fba-safe-live}讨论了为何我们能够适度地希望它们这样做。然而,当安全依赖于用户可配置的参数时,总有可能人们会把它们设置错了。

甚至当人们正确设置了{\quorum}切片且SCP确保了安全性,安全性本身并未踢出可能在联邦系统中的其他的安全问题。例如,在一个金融市场上,被广泛信任的节点可能会改变它们在网络中的地位来获取一些信息,这些信息可能会被用于超前交易或者其他不道德的行为。

拜占庭节点可能会在SCP的输入端尝试过滤一些交易而另一方面产生正确的输出。如果良性行为的节点接受所有的交易,结合函数取所有交易的并;而存在完好的节点,那么这种过滤将不会成功地让受害者交易以概率1被阻塞而是表现出延迟。

尽管SCP的安全性是最优的,它的性能和通信延迟不是。
\section{总结}\label{sec:summary}

拜占庭协商已经长期使得分布式系统获得高效的共识、标准化的加密安全性以及设计可信参与者的灵活性。更近一些发生的是,比特币引入了去中心化共识的革命性观念,带来了许多系统和研究的挑战。本文介绍了联邦拜占庭协商(FBA),一种保留传统拜占庭协商带来的便利且能达成去中心化共识的模型。FBA和以往的拜占庭共识系统的关键不同在于FBA从参与者的个人的信任决策生成{\quorum},使得类似于互联网的有机生长模型成为可能。恒星共识协议(SCP)是一种最佳的对抗恶性行为参与者使得系统恢复的构建方法。
\section{致谢}

Jed McCaleb激发了这一工作的灵感并提供的反馈、术语使用建议,同时他还帮忙考虑了多种假想情形。Jessica Collier协助撰写了本文。Stan Polu完成了SCP第一个实现并在过程中提供了非常宝贵的修改、建议、简化和反馈。Jelle van den Hooff提供了在{\quorum}交和联邦投票章节上的重新调整结构的关键思路,同时还包括术语、组织和演讲等其它方面上重要的建议。Nicolas Barry在他实现这一协议的时候发现了多处错误,并指出了必要的澄清。Ken Birman, Bekki Bolthouse, Joseph Bonneau, Mike Hamburg, Graydon Hoare, Joyce Kim, Tim Makarios, Mark Moir, Robert Morris, Lucas Ryan, 以及Katherine Tom辛勤地校对论文草稿,找出了多处错误以及会引起困惑的地方,同时还提供了有益的建议。Eva Gantz提供了实用的写作动机和引用文献。Winnie Lim在图表上提供了指导。Reddit社区和Tahoe-LAFS小组指出了早期版本的SCP中审查的脆弱性,带来了增强版的提名协议。最后,作者还感谢整个恒星团队提供的支持、反馈和鼓励。
\section{免责声明}
Mazi{\`e}res教授是作为带薪顾问为本文作出贡献的,这并非他在斯坦福大学的职责。
%\input{tex/appA-glossary}
%\printglossaries

\bibliographystyle{abbrv}
\bibliography{ref}


\end{document}
